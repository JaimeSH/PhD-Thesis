\chapter{Conclusions and Future Work}
\label{chapter:conclusions-and-future-work}

The reader can find in the following Sections conclusions about the different
parts that compose the proposed method in this thesis, about its implementation
and about the results obtained in the experiments involving the implementation
described in the document. Moreover, the second and last Section in this Chapter
proposes a variety of works that can be done in the future in order to improve
the proposed method and that can support the findings shown in the document.

\section{Conclusions}
\label{section:conclusions}

This thesis proposes a novel method for forecasting and interpreting financial
markets. The method is robust and allows versatility in how the
different composing parts can interact without losing a clear set of
constraints. For instance, the proposed method forces the implementation of a
preprocessing algorithm for the financial market price data, but the method is
flexible enough to allow any preprocessing algorithm that gives as a result a
real number that can be fed to the agents' fuzzy systems. Furthermore, the fuzzy
systems used to manage the agents' decisions can be changed to any rule-based
system, or even any mapping algorithm that can determine a decision based on a
set of inputs, such as a neural network -- although some systems could cause the
proposed method to lose its interpretability.

Alternatively, if the user decides to use the parts proposed, these parts are
still customizable: the preprocessing algorithm can be configured to give as a
result multiple price areas of different sizes, which can then be served as
input to the intuitionistic fuzzy system, which can take virtually any number of
inputs. The complexity of the intuitionistic fuzzy system can be adjusted by
changing the type and number of membership functions and non-membersip
functions. The defuzzification method for the intuitionistic fuzzy system can
also be changed to other than the default centroid-based method. If the user
desired, the intuitionistic fuzzy system could be reduced to a traditional fuzzy
system by lowering the hesitancy in the membership functions to 0.

The proposed method was only tested with five financial markets in Chapter
\ref{chapter:experiments-and-results}, but the system should be able to adapt to
any kind of financial market, or even time series that do not represent
financial markets, provided that the preprocessing algorithm is meaningful for
the dataset's nature.

The optimization of the proposed method can be performed using a variety of
algorithms. At certain points during the development of the thesis, particle
swarm optimization and differential evolution were considered to be used as the
optimization algorithms for the proposed method, but the use of a genetic
algorithm was preferred due to its simplicity. Furthermore, the fuzzy systems
could be manually adjusted in order to prove particular hypotheses about a
financial market's behavior due to its human-readable representation.

The implementation and its experiments described in Chapters
\ref{chapter:implementation} and \ref{chapter:experiments-and-results},
respectively, demonstrate how the system can be adjusted in order to obtain
systems that vary in their generalization capabilities. A low number of
agents with low number of rules could not simulate a good representation of the
real market prices; high number of agents with high number of rules sometimes
performed well and sometimes performed poorly due to the relatively low number
of generations that was set for the genetic algorithm; and most of the times,
the models with a moderate number of agents and rules were able to generalize
the markets behaviors. These Chapters also demonstrate how the generated models
can be used to interpret the nature of the markets, although the interpretations
presented in the experiments lack some robustness and their redaction could be
improved.

\section{Future Work}
\label{section:future-work}

The conclusions provided in the last Section give a notion about the
different directions the proposed method could take in the future. For instance,
the preprocessing algorithm can be configured to provide a greater number of
inputs to the agents' rule systems; the list of beliefs could be extended in
order to increase the number of ways an agent can perceive a market; the range
of the possible belief values could be modified; the size of the resulting price
areas can be changed; the way the weight for the resulting areas is determined
could be changed to, for example, one that incorporates a way to represent
degradation according to time of the price that generated certain retracements,
i.e. associate lesser weights to older price retracements. Additionally, the
preprocessing algorithm could be changed completely.

It was considered to provide a comparison between the use of intuitionistic
fuzzy systems and traditional type-1 fuzzy systems for the agents' rule
systems. This can be done in a future work. All the fuzzy systems in the
proposed method are limited to the use of Gaussian membership and non-membership
functions; combinations of different types of functions could be tested to see
if the performance and interpretability of the system improve.

The optimization algorithm used in the proposed method is another factor that
can be studied in future works. A current limitation of the genetic algorithm is
that it is limited to the initial population's genes, plus some mutations
occurring from time to time. Perhaps meta-heuristics that are suited to handle
continuous search spaces would be better suited for the optimization of the
agent-based models.

Lastly, even though a sufficiently large number of experiments were performed in
order to not be able to include all of them in the thesis document, more
experiments still need to be performed to further support the capacities of the
method. Particularly, the number of generations used for the genetic algorithm
is considerably low. Experiments where greater numbers of generations for a
genetic algorithm or any other meta-heuristic should yield far better results
than the ones presented in this document.
