\chapter{Proposed Method}
\label{chapter:proposed-method}

This Chapter describes each of the modules that integrate the method used to
create the trading strategy proposed in this work and how their integration is
performed.

\section{Preprocessing a Financial Market using Retracements}
\label{section:preprocessing-a-financial-market-using-retracements}

Many works that propose methods for financial market forecasting or modelling do
not preprocess the time series that represent a financial market \cite{}. As a
consequent, some of the models that try to simulate these markets have an additional
complexity layer to deal with, similarly to the difficulty that a neural network
would encounter at trying to do facial recognition to un-processed images. For
this reason, a facial recognition algorithm needs to be fed images that have
been rotated, scaled down, and reduced in noise by using image processing
algorithms \cite{}.

For financial market preprocessing, an option is to use technical indicators:
mathematical calculations that are applied to time series that represent the
historical prices of a financial market. As an example of a technical
indicator's use, a moving average -- a series of averages of different subsets
of the full data set -- can remove the noise in a market by creating a smooth
line that represents each of the data points in the time series, as can be seen in Figure
\ref{figure:moving-average-noise}. Other technical indicators can provide other
types of insight about a market, such as the market's volatility, support and
resistance levels, and overbought and oversold levels, among others \cite{}.

\begin{figure}
\caption{Using a moving average to remove the noise from a financial market time
series} \centering
\includegraphics[width=1.0\textwidth]{img/moving-average-noise.png}
\label{figure:moving-average-noise}
\end{figure}



\begin{itemize}
\item Each agent will have its own version of the preprocessed market
\end{itemize}

\section{Using Agents to Represent Traders in a Financial Market}
\label{section:using-agents-to-represent-traders-in-a-financial-market}

\section{Representing the Agents' Rules as Intuitionistic Fuzzy Systems}
\label{section:representing-the-agents-rules-as-intuitionistic-fuzzy-systems}

\subsection{Indeterminacy or Hesitancy}
\label{subsection:indeterminacy-or-hesitancy}

\section{Generation of an Agent-Based Model}
\label{section:generation-of-an-agent-based-model}

\section{Using the Agent-Based Model to Generate Insights about a Financial
Market}
\label{section:using-the-agent-based-model-to-generate-insights-about-a-financial-market}

\section{Using the Agent-Based Model to create a Trading Strategy}
\label{section:using-the-agent-based-model-to-create-a-trading-strategy}
