\chapter{Implementation}
\label{chapter:implementation}

\section{Preprocessing a Financial Market using Retracements}
\label{section:preprocessing-a-financial-market-using-retracements:implementation}

The author of this thesis has worked with a variety of technical indicators for
trading a number of financial markets, and has found that support and
resistance-based technical indicators can be used to successfully trade in a
financial market without the use of any other indicators. For this reason, the
proposed method uses a preprocessing process based on this type of technical
indicators. Although it is a subjective reason, it is paramount to note that the
proposed method can easily be adapted to any other preprocessing method; the
preprocessing stage in this work is used mainly in the hopes of facilitating the
agents the interpretation of the market, and thus facilitate the creation of a
model that simulates the market.

\begin{itemize}
\item Each agent will have its own version of the preprocessed market
\end{itemize}

\section{Using Agents to Represent Traders in a Financial Market}
\label{section:using-agents-to-represent-traders-in-a-financial-market:implementation}

\section{Representing the Agents' Beliefs as Intuitionistic Fuzzy Systems}
\label{section:representing-the-agents-beliefs-as-intuitionistic-fuzzy-systems}

\section{Generating the Agent-Based Model}
\label{section:generating-the-agent-based-model}

\section{Extracting Insights about a Financial Market}
\label{section:extracting-insights-about-a-financial-market}
