% $Log: abstract.tex,v $
% Revision 1.1  93/05/14  14:56:25  starflt
% Initial revision
% 
% Revision 1.1  90/05/04  10:41:01  lwvanels
% Initial revision
% 
%
%% The text of your abstract and nothing else (other than comments) goes here.
%% It will be single-spaced and the rest of the text that is supposed to go on
%% the abstract page will be generated by the abstractpage environment.  This
%% file should be \input (not \include 'd) from cover.tex.

El entender el comportamiento de los mercados financieros es importante para determinar el estado de la economía de un país. Las herramientas actuales para entender estos comportamientos son variadas y usualmente pueden ser asignadas a una de dos categorías: análisis fundamentalista o técnico. El análisis fundamentalista se basa en analizar cualquier factor que pueda afectar el valor de un activo, tales como condiciones financieras o la administración de una compañía, mientras que el análisis técnico se enfoca exclusivamente en analizar los movimientos de los precios de un mercado. Esta tesis presenta un método de predicción para mercados financieros basado en análisis técnico. El método puede ser usado como una estrategia de intercambio y como una herramienta para entender el comportamiento de un mercado financiero y sigue una arquitectura basada en agentes donde las reglas de los agentes están basadas en lógica difusa. Específicamente se usan sistemas difusos intuicionistas que le permiten a los agentes modelar tanto la incertidumbre como la indecisión en un mercado. Para poder modelar la percepción de los agentes, los precios de un mercado son preprocesados usando un algoritmo basado en un indicador técnico basado en retrasos. Una implementación del método propuesto demuestra la versatilidad del sistema, ya que los modelos generados pueden llevar a estrategias de intercambio rentables, así como modelos que pueden ser interpretados por un ser humano para obtener posibles explicaciones sobre el comportamiento de un mercado.

\clearpage
\section*{Abstract}

Understanding financial markets is paramount in order to determine the well-being of a country's economy. The currently available tools to obtain such understandings are varied and can usually fall under one of two categories: fundamental or technical analysis. Fundamental analysis relies on analyzing any factor that can affect an asset's value, such as financial conditions and company management, whereas technical analysis relies exclusively on analyzing a market price movements. This thesis presents a financial market forecasting method based on technical analysis. The method can be used as a trading strategy and as a tool for understanding the behavior of a financial market and follows an agent-based architecture where the agent rules are based on fuzzy logic. Specifically, intuitionistic fuzzy systems are used to allow the agents model the uncertainty and hesitancy in a market. In order to model the agents' perception, the market price data is preprocessed using an algorithm based on a retracements technical indicator. An implementation of the proposed method shows the versatility of the system, as the generated models can provide profitable trading strategies, as well as models that can be interpreted by a human being to obtain possible explanations for a market's behavior.